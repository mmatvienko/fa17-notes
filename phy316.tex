\documentclass{article}
    \usepackage[utf8]{inputenc}
    \usepackage[english]{babel}
    \usepackage[]{amsthm} %lets us use \begin{proof}
    \usepackage{amsmath}
    \usepackage{gensymb}   
    \usepackage{circuitikz} 
    
    \usepackage{mathrsfs}   
\DeclareFontFamily{U}{calligra}{}
\DeclareFontShape{U}{calligra}{m}{n}{<->callig15}{}
\newcommand{\calE}{{\!\!\text{\usefont{U}{calligra}{m}{n}E}\,\,}}

    \usepackage[]{amssymb} %gives us the character \varnothing
    
    \title{PHY 316M}
    \author{Marc Matvienko}
    %This information doesn't actually show up on your document unless you use the maketitle command below
    
    \begin{document}
    \maketitle %This command prints the title based on information entered above
    
    \section{Capacitors}
    $C = \lvert\frac{Q}{V}\rvert$

    \section{Current}
    Is the flow of charge in on direction.
    Current: $$I=\frac{\Delta Q}{\Delta t} =\frac{\delta Q}{\delta t}$$
    Current Density (current per unit area): $J = \frac{I}{A}$\\
    $n = \text{charge carriere density}$, $q = \text{charge per carrier}$, $v_d = \text{drift velocity}$
    This can give us, \boxed{$$J = nqv_d$$}
    
    \subsection{Ohm's Law}
    Usually we see Ohm's law in different forms, i.e. for a particular chunk.\\
    Consinder some block with volume $A \cdot l$, some source of energy(battery) forces current thorugh by applying an electric field.\\
    For a uniform electric field: \boxed{$$V = E\cdot l$$}
    
    \begin{align*}
        \begin{split}
            J &= \frac{I}{A} \\
            &= \sigma E \\
            &= \sigma \frac{V}{l}
        \end{split}
    \end{align*}
    $$\boxed{$$\frac{V}{I} = \frac{1}{\sigma} \frac{l}{A} = \rho \frac{l}{A} = R = \text{resistance}$$}$$
    \boxed{$$\frac{V}{I} = R$$}
    Resistance is not resistivity (opisiton of current flow of a particular material[$\rho$])
    Ohmic material is a material that has a constant slope on Voltage to Current graph. Most common materials like copper behave like this.
    %Section and subsection automatically number unless you put the asterisk next to them.
    \paragraph{Example} The resistivity of nichrome wire( heaters, toasters) is $1.5 \times 10^{-6}\Omega m$. 
    If a household voltage of $115V$ is applied acros a $0.2mm$ radius write, $1.0m$ long, what current flows?
    \paragraph{}$R = \rho \frac{l}{A} = \rho \frac{l}{\pi r^2} = \frac{1.5 \times 10^{-6}\Omega m \cdot 1.0 m}{(\pi (2\times 10^{-4}m)^2))} = 11.9 \Omega$    
   
    \subsection{Model for electric conduction}
    \begin{itemize}
        \item electron unergo many rapid ocllision when  $E = 0$
        \item when $E \neq 0$, the electrons accelrate between collisions
        \item $F = ma = qE => a = \frac{qE}{m}$
        \item $v = v_0 + at= v_0 + \frac{qE}{m}t $
    \end{itemize}
    Let $\tau = \text{average collision time} = R\cdot C$\\
    the  $v_d = v_{avg} = <v_0> + \frac{qE}{m}\tau$\\
    so, $J = nqv_d = nq = \frac{qE}{m}\tau = \sigma E$\\
    so conducitvity $\sigma = \frac{nq^2\tau}{m}$\\
    Called the Drude model or free electron mode\\
    $$\boxed{$$\sigma = \frac{nq^2 \tau}{m}$$}$$
    $$\boxed{$$\frac{1}{\sigma} = \rho$$}$$
    \paragraph{Example} Assume for copper that each atom donates one free electron. What is the average time between collision for electrons in copper.?
    \\Given: 
    \begin{itemize}
        \item Density$ = 8.98\frac{g}{cm^3}$
        \item Atomic Weight $ = 63.54\frac{g}{mole}$
        \item $\rho = 1.7 \times 10-^{-8}\Omega\cdot m$
    \end{itemize}
    $\tau = \frac{m}{nq^2\rho}=\frac{9.14 \times 10^{-31}kg}{(8.5\times10^22 ) (1.6\times10^{-19})^2 1.7\times10^{-8}\Omega m} = 2.5\times10^{-14}s$
    \subsection{Temerature Dependence of resistivity}
    \begin{itemize}
        \item resistivities tabulated for 20 \degree celsius
        \item for metals, $\rho$ is higher and T is higher
        \item $\alpha = \text{linear temprature coefficient}$
        \item over some range, \boxed{$$\rho = \rho_0 (1+ \alpha(T-T_0))$$}
    \end{itemize}
    As T increases, the scattering time decreases due to collisions with vibrating atoms\\
    At higher temperatures the $\rho$ to temprature graph  is linear. But at the beginning there is residual resistvity due to impurities.
   \paragraph{Semiconductors} The number of carriers decreases as the temperature decreases, this means that all the electrons are sticking to their atoms.
   \section{Resistors in Series and Parallel}
    Circuit symbol: 
    \begin{circuitikz}\draw
        (0,0)to[resistor, l=$R$] (4,0)
    ;\end{circuitikz}
    \subsection{Resistors in series} For resistors in series the resistivities add
    $$R_{tot} = R_1 + R_2 + ...$$
    $$R_{tot} = \frac{V}{I} = \frac{V_1}{I_1} + \frac{V_2}{I_2} = R_1 + R_2$$ 
    The current ( I ) is the same everywhere too.\textbf{ Resistors don't add in parallel. Capacitors do.}
    $$V = V_1 + V_2$$ 
    \\The voltage divider $$V_1 = I\cdot R_1 = \frac{V}{R_1 + R_2}\cdot R_1$$
    \subsection{Resistors in parallel}
    For resisitors in parallel halve the resistance if two exact resistors are put in parallel    
    \begin{itemize}
        \item In parallel have the same voltage across each element
        \item In parallel also the current divides among branches
    \end{itemize}
    $$I = I_1 + I_2 = \frac{V}{R_1} + \frac{V}{R_2} => R_{tot} = \frac{V}{I} = \frac{1}{\frac{1}{R_1}+\frac{1}{R_2}}$$
    \paragraph{Superconductors} Electrons pair up and when electron jumps to the lattice another electrons pulls it right back.
    \subsection{Resistors Disipate Energy}
    Electrons undergo collisions, and give up energy as heat. A steady release of current ($I$) causes a steady realease of energy.
    $$\Delta U = \Delta QV$$
    Better to discess the rate, which is really known as:
    $$\text{Power}=P=\frac{\Delta U}{\Delta t} = \frac{\Delta U}{\Delta t} V = IV$$
    $$\boxed{$$P = IV = I^2R = \frac{V^2}{R}$$}$$
    This power is also known as Joule heating.
    \paragraph{Putting this into practice:}many circuits can be analyzed with just Ohm's Law and Resistance.
    \begin{itemize}
        \item What is the total power delivered? 
        \item What is the power dissipated in each R?
    \end{itemize}
    \begin{circuitikz}\draw
        (0,-2)to[battery, l=$12V$] (0,1)
        (0,1) to (3,1)
        (3,1)to[resistor, l=$2\Omega$](3,-1)
        (2, -1)to(4,-1)
        (2,-1)to[resistor, l=$3\Omega$](2,-3)
        (4,-1)to[resistor, l=$6\Omega$](4,-3)
        (2,-3)to(4,-3)
        (3,-3)to(3,-4)
        (3,-4)to(0,-4)
        (0,-4)to(0,-2)
    ;\end{circuitikz}
    $P_{\text{dissapated in } 3\Omega} = \frac{V_3^2}{R} = \frac{(6V)^2}{3\Omega}=12W$
    \subsection{Direct Current Circuits}
    Real battery is an ideal $\calE$MF plus intended resistance\\
    \begin{circuitikz}\draw
        (0,0)
        node[label={left:Batteries are a source of voltage}] {}
        to[battery, l=$+$](2,0)
    ;\end{circuitikz}
    \\Source of voltage = "electromotive force" = $\calE$MF= $\calE$
    $$\boxed{$$V = \calE - Ir$$}$$
    "Open-circuit voltage", where $I=0$
    
    \paragraph{Analysis of circuits}
    Any circuit can be analyzed with Kirchhoffs Rules:
    \begin{itemize}
        \item Junction Rule, algebraic sum of currents into a junction = sum of currents $\Sigma i_{in} = \Sigma i_{out}$
        \item Loop Rule, algebraic sum of voltages around any closed loop is zero. Where voltage rises are positive (- to +) and votlage drops are negative (+ to -)
    \end{itemize}
    For resistors the current directino determines voltage drop(negative)

    \paragraph{Example}
    \begin{itemize}
        \item What is the current in the $6 \Omega$ resisitor?
        \item Is it flowing up or down?
    \end{itemize}
    \begin{circuitikz}\draw
        (0,-2)to[battery,l=$12V$](0,2)
        (0,2)to[resistor, l=$4\Omega$](4,2)
        (4,2)to[resistor, l=$6\Omega$](4,-2)
        (4,2)to[resistor, l=$2\Omega$](8,2)
        (8,2)to[battery,l=$12V$](8,-2)
        (0,-2)to(8,-2)
    ;\end{circuitikz}\\
    \begin{itemize}
        \item [\textbf{(I)}]Junction at A: $i_1 = i_2 + i_3$
        \item [\textbf{(II)}] Loop A: $12V - i_1(4\Omega) - i_3(6 \Omega) = 0$
        \item [\textbf{(III)}]Loop B: $I_3(6\Omega) - i_2(2\Omega) + 12V = 0$
    \end{itemize}
 
    $i_3 = \frac{-12V}{22\Omega} = -\frac{6}{11}A$
    \paragraph{2 other techniques:}
    \begin{itemize}
        \item[i)] same, except use ficticious ``loop currents''
        \item[ii)] Source suppressing - can look at effects of sources seperately
    \end{itemize}
    \paragraph{Next: Circuits with Capacitors} will see time-depedent behavior
    Before ``transient phenomena'', look at \textit{Steady-state}: ``after a long time''. The capacitor starts acting like the current is 0.

    \section{Magnetism}
    We saw: $$F = qv\frac{\mu_0 I}{2\pi r} = 0$$
    for $\vec{v}$ tangent to circle.
    where $\mu_0 = 4\pi \times 10^{-7} \frac{Ns^2}{C^2}$\\
    Rewrite this as $$\boxed{$$\vec{F} = q\vec{v}\times \vec{B}$$}$$
    where \boxed{$$B = \frac{\mu_0 I}{2 \pi r}$$} magentic field due to a long wire where direciton of $\vec{B}$ is tangent to circle.
    \\If also electric force, combination of electric and magnetic force is called ``Lorentz Force'' $$\vec{F} = q\vec{E} + q\vec{v}\times \vec{B}$$
    $\vec{F} = q\vec{v}\times \vec{B}$
    \begin{itemize}
        \item Like $\vec{E}, \vec{F}$ opposite for $+q, -q$
        \item Depends on $\vec{v}: \vec{F} = 0 \text{ for } \vec{v} = 0$
        \item Depends on angle : \boxed{$$\lvert\vec{F}\rvert = \lvert q \rvert v B sin\Theta$$}\\
        notice $\vec{F} = 0$ for $\vec{v} \parallel \vec{B}$
        \item Direction $\vec{F} = q\vec{v}\times\vec{B}$ given by right hand rule
    \end{itemize}
    Magnetic field does no work because the magnetic force is perpendcular to displacement.\\
    So $\vec{B}$ changes direction of $\vec{v}$ but not its magnitude ($KE = \frac{1}{2}mv^2$)\\
    Unit of B = $\frac{Ns}{Cm} = \text{tesla} = T$ also gauss $= G = 10^{-4} T$
    \subsection{Evaluating Cross Products}
    \begin{itemize}
        \item [1.] Get general equation by expaning determinant 
        $\begin{vmatrix} i & j & k \\ v_x & v_y & v_z \\ B_x & B_y & B_z\end{vmatrix} = $\\
            $\vec{v}\times \vec{B} = i(v_yB_z-v_zB_y) - j(v_xB_z - v_zB_x) + k(v_xB_y-v_yB_x)$
        \item [2.] Just multiply out.\\
        know that $i \times j = k$ and that $j \times i = -k$
        \item [5.] Magnetic field does no work
    \end{itemize}
    \subsection{Two ways to Find $\vec{B}$}
    \begin{enumerate}
        \item Ampere's Law: for high symmetry
        \item Biot Savart Law
    \end{enumerate}
    In general we will only look at: center of arcs and circles, or due to straight segments, or on axis of loop.
    \subsubsection{$\vec{B}$ at the center of a circle}
    $$d\vec{B} = \frac{\mu_0}{4\pi}\frac{Id\vec{s}\times \hat{r}}{r^2}$$
    $$B = \int dB = \int \frac{\mu_0 I}{4 \pi r^2}ds = \frac{\mu_0 I}{2R}$$
    Since $r=R$, the integral is constant and is easy to integrate with respect to s.
    \paragraph{Also:}for a fraction f of a circle \boxed{$$\vec{B} = fB_{\text{full circle}}$$}
    
    \subsubsection{$\vec{B}$ near a finite straight wire}
   
    \begin{align*}
        \begin{split}
            d\vec{B} &= \frac{\mu_0 I}{4\pi}\frac{d\vec{s}\times \hat{r}}{r^2}\\
                &= \frac{\mu_0 I}{4\pi} \frac{ds sin\theta}{r^2}
        \end{split}
    \end{align*}

    \begin{enumerate}
        \item $sin\theta = sin\theta ' =cos\phi$
        \item $cos\phi = \frac{R}{r} \rightarrow r \frac{R}{cos\phi}$
        \item $tan\phi = \frac{s}{R} \rightarrow s =R\cdot tan\phi \rightarrow \frac{R}{cos^2\phi d\phi}$
    \end{enumerate}

    \begin{align*}
        \begin{split}
            B &= \int dB \\
            &= \int \frac{\mu_0I \frac{R}{cos^2\phi}d\phi cos\phi}{4\pi (\frac{R}{cos\phi})^2} \\
            &= \frac{\mu_0 I}{4\pi R}\int_{-\phi_1}^{+\phi^2}cos\phi d\phi \\
            &= \frac{\mu_0 I}{4\pi R}sin \phi \rvert_{-\phi_1}^{+\phi^2} \\
            &= \frac{\mu_0 I}{4\pi R}[sin \phi _2 + sin\phi _1]
        \end{split}
    \end{align*}

    \subsubsection{B field due to a square loop of side a}
    $$B_{loop}$$
    $$\frac{2\sqrt{2}\mu_0 I}{\pi a}$$

    \subsubsection{B on axis of loop}
    Off axis compnents cancel around circle. \\
    $\phi$ is the angle at the bottom right of triangle formed by circle and axis
    \begin{align*}
            r &= \sqrt{x^2 + R^2} && \text{Using pythagorean theorem}\\
            sin\phi &= \frac{R}{\sqrt{x^2 +R^2}} && \text{So we can define }sin\phi
    \end{align*}
    We only have to integrate along the x compnents
    \begin{align*}
            dB_x &=dB sin\phi \\
            &= \frac{\mu_0 I ds}{4\pi r^2}sin\phi
    \end{align*}
    We can say that $\int \,ds = 2\pi R $ since the radius is constant\\
    \begin{align*} 
        B &= \int \frac{\mu_0 I}{4\pi r^2}sin\phi \,ds\\
        &= \frac{\mu_0 I R}{4 \pi (x^2+R^2)^{3/2}}\int \,ds\\
        &= \frac{\mu_0 I R^2}{2 (x^2+R^2)^{3/2}}     
    \end{align*}
    
    Also, consider yourself very close to the field, $ x >> R ~ \frac{1}{x^3}$ dipole field
    \subsection{Motion in a uniform B}
    $$ \lvert \vec{F} \rvert = \lvert q\vec{v} \times \vec{B} \rvert = qvB $$
    $\vec{F}$ is $\bot$ to $\vec{v}$ centripetal with accelarations $a_r = \frac{v^2}{r}$
    We saw that 
    \begin{align*}
        B_{\text{long wire}} &= \frac{\mu_0 I}{2\pi r}\\
        B_{\text{solonoid}} &= \mu_0 n I \\
        B_{\text{loop center}} &= \frac{\mu_0 I}{2R}
    \end{align*}
    $\text{angular velocity } \omega = ?$
    $$r = \frac{mv}{qB}$$
    \paragraph{Example: Mass Spectrometer}
    \textit{Note: Electrons travel in a semi-circle in a spectrometer}
    Electrons are accelarated through potential of $10^3 V$ ("a 1 keV electron"). 
    They enter a region of uniform $B = 10^{-2}T$. What is the distance they are displaced?
    $$x = 2r = 2 \frac{mv}{qB}$$\\
    Know $m,q,B$ need to find $v$\\
    \begin{align*}
        \Delta KE &= \Delta PE \rightarrow \frac{1}{2}mv^2 = eV_0= \sqrt{\frac{2eV}{m}} \\
        x &= 2\frac{m}{eB}\sqrt{\frac{2eV_0}{m}} = \frac{2}{B}\sqrt{\frac{2mV_0}{e}} = \frac{2}{(10^{-2}T)}\sqrt{\frac{2(9.11\times 10^{-31})(10^3V)}{(1.6\times 10^{-19}C)}}        
    \end{align*}
    \\In practice however, we are given v,B,x to get ($\frac{q}{m}$)

    \paragraph{Example: Velocity Selector}region of crossed $\vec{E} + \vec{B}$\\
    All we have to consider is $qE \textit{ -vs- } qvB$\\
    if $qE = qvB$ then \boxed{$$v = \frac{E}{B}$$}

    \subsection{Force on a current}
    Consider positive charge travelling along x axis with $v_0$ with a $-\vec{B}\hat{y}$
    \begin{align*}
        F_{\text{on wire}} &= q\vec{v}\times\vec{B}\\
        &= \frac{1}{n}\vec{j}\times\vec{B}\\
       F_{\text{on wire}}&=\text{(\# charges)}F_{\text{on 1 charge}}\\
       &=(nAl)\frac{1}{n}\hat{\vec{j}}\times\hat{\vec{B}}\\
       &=lA\vec{j}\times\vec{B}\\
       &=I\vec{l}\times\vec{B}
    \end{align*}
    \paragraph{Example}
    Net force on a current loop in a uniform $\vec{B}$ is \textbf{zero}. 
    This is true for any loop.
    \paragraph{Example} A current I flows from the origin to $(x,y,z) = (1m,1m,0)$ and then straight to $(2m,0,0)$. In a uniform field of $\vec{B} = 5T\hat{i}$
    \begin{align*}
        \vec{F}_1 &= (10N)(-\hat{j} + \hat{i})\\
        \vec{F}_2 &= (10N)(-\hat{j}-\hat{i})\\
        \vec{F} &= (20N)-\hat{j}
    \end{align*}

    \paragraph{Example} Force on a wire segment due to a large $\parallel$ wire
    \begin{align*}
        B &= \frac{\mu_0 I}{2\pi r}\\
        \vec{F} &= I_2\vec{l}\times \frac{\mu_0 I_1}{2\pi r}\\
        &= I_2\times \frac{l \mu_0 I_1}{2\pi r}\\
        \frac{\text{force}}{\text{length}} &= \frac{\mu_0 I_1 I_2}{2\pi r}
    \end{align*}
    \paragraph{Definition:}Magnetic Moment\\
    Dipole moment ($\mu = IA$) due to a magnetic loop.
\end{document}