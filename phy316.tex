%This is my super simple Real Analysis Homework template

\documentclass{article}
    \usepackage[utf8]{inputenc}
    \usepackage[english]{babel}
    \usepackage[]{amsthm} %lets us use \begin{proof}
    \usepackage{amsmath}
    \usepackage{gensymb}   
    \usepackage{circuitikz}   
    \usepackage{mathrsfs}   
\DeclareFontFamily{U}{calligra}{}
\DeclareFontShape{U}{calligra}{m}{n}{<->callig15}{}
\newcommand{\calE}{{\!\!\text{\usefont{U}{calligra}{m}{n}E}\,\,}}

    \usepackage[]{amssymb} %gives us the character \varnothing
    
    \title{PHY 316M}
    \author{Marc Matvienko}
    %This information doesn't actually show up on your document unless you use the maketitle command below
    
    \begin{document}
    \maketitle %This command prints the title based on information entered above
    
    \section{Capacitors}
    $C = \lvert\frac{Q}{V}\rvert$

    \section{Current}
    Is the flow of charge in on direction.
    Current: $$I=\frac{\Delta Q}{\Delta t} =\frac{\delta Q}{\delta t}$$
    Current Density (current per unit area): $J = \frac{I}{A}$\\
    $n = \text{charge carriere density}$, $q = \text{charge per carrier}$, $v_d = \text{drift velocity}$
    This can give us, \boxed{$$J = nqv_d$$}
    
    \subsection{Ohm's Law}
    Usually we see Ohm's law in different forms, i.e. for a particular chunk.\\
    Consinder some block with volume $A\dot l$, some source of energy(battery) forces current thorugh by applying an electric field.\\
    For a uniform electric field: \boxed{$$V = El$$}
    
    \begin{align*}
        \begin{split}
            J &= \frac{I}{A} \\
            &= \sigma E \\
            &= \sigma \frac{V}{l}
        \end{split}
    \end{align*}
    $$\boxed{$$\frac{V}{I} = \frac{1}{\sigma} \frac{l}{A} = \rho \frac{l}{A} = R = \text{resistance}$$}$$
    \boxed{$$\frac{V}{I} = R$$}
    Resistance is not resistivity (opisiton of current flow of a particular material[$\rho$])
    Ohmic material is a material that has a constant slope on Voltage to Current graph. Most common materials like copper behave like this.
    %Section and subsection automatically number unless you put the asterisk next to them.
    \paragraph{Example} The resistivity of nichrome wire( heaters, toasters) is $1.5 \times 10^{-6}\Omega m$. 
    If a household voltage of $115V$ is applied acros a $0.2mm$ radius write, $1.0m$ long, what current flows?
    \paragraph{}$R = \rho \frac{l}{A} = \rho \frac{l}{\pi r^2} = \frac{1.5 \times 10^{-6}\Omega m \cdot 1.0 m}{(\pi (2\times 10^{-4}m)^2))} = 11.9 \Omega$    
   
    \subsection{Model for electric conduction}
    \begin{itemize}
        \item electron unergo many rapid ocllision when  $E = 0$
        \item when $E \neq 0$, the electrons accelrate between collisions
        \item $F = ma = qE => a = \frac{qE}{m}$
        \item $v = v_0 + at= v_0 + \frac{qE}{m}t $
    \end{itemize}
    Let $\tau = \text{average collision time}$\\
    the  $v_d = v_{avg} = <v_0> + \frac{qE}{m}\tau$\\
    so, $J = nqv_d = nq = \frac{qE}{m}\tau = \sigma E$\\
    so conducitvity $\sigma = \frac{nq^2\tau}{m}$\\
    Called the Drude model or free electron mode\\
    $$\boxed{$$\sigma = \frac{nq^2 \tau}{m}$$}$$
    $$\boxed{$$\frac{1}{\sigma} = \rho$$}$$
    \paragraph{Example} Assume for copper that each atom donates one free electron. What is the average time between collision for electrons in copper.?
    \\Given: 
    \begin{itemize}
        \item Density$ = 8.98\frac{g}{cm^3}$
        \item Atomic Weight $ = 63.54\frac{g}{mole}$
        \item $\rho = 1.7 \times 10-^{-8}\Omega\cdot m$
    \end{itemize}
    $\tau = \frac{m}{nq^2\rho}=\frac{9.14 \times 10^{-31}kg}{(8.5\times10^22 ) (1.6\times10^{-19})^2 1.7\times10^{-8}\Omega m} = 2.5\times10^{-14}s$
    \subsection{Temerature Dependence of resistivity}
    \begin{itemize}
        \item resistivities tabulated for 20 \degree celsius
        \item for metals, $\rho$ is higher and T is higher
        \item $\alpha = \text{linear temprature coefficient}$
        \item over some range, \boxed{$$\rho = \rho_0 (1+ \alpha(T-T_0))$$}
    \end{itemize}
    As T increases, the scattering time decreases due to collisions with vibrating atoms\\
    At higher temperatures the $\rho$ to temprature graph  is linear. But at the beginning there is residual resistvity due to impurities.
   \paragraph{Semiconductors} The number of carriers decreases as the temperature decreases, this means that all the electrons are sticking to their atoms.
   \section{Resistors in Series and Parallel}
    Circuit symbol: 
    \begin{circuitikz}\draw
        (0,0)to[resistor, l=$R$] (4,0)
    ;\end{circuitikz}
    \subsection{Resistors in series} For resistors in series the resistivities add
    $$R_{tot} = R_1 + R_2 + ...$$
    $$R_{tot} = \frac{V}{I} = \frac{V_1}{I_1} + \frac{V_2}{I_2} = R_1 + R_2$$ 
    The current ( I ) is the same everywhere too.\textbf{ Resistors don't add in parallel. Capacitors do.}
    $$V = V_1 + V_2$$ 
    \\The voltage divider $$V_1 = I\cdot R_1 = \frac{V}{R_1 + R_2}\cdot R_1$$
    \subsection{Resistors in parallel}
    For resisitors in parallel halve the resistance if two exact resistors are put in parallel    
    \begin{itemize}
        \item In parallel have the same voltage across each element
        \item In parallel also the current divides among branches
    \end{itemize}
    $$I = I_1 + I_2 = \frac{V}{R_1} + \frac{V}{R_2} => R_{tot} = \frac{V}{I} = \frac{1}{\frac{1}{R_1}+\frac{1}{R_2}}$$
    \paragraph{Superconductors} Electrons pair up and when electron jumps to the lattice another electrons pulls it right back.
    \subsection{Resistors Disipate Energy}
    Electrons undergo collisions, and give up energy as heat. A steady release of current ($I$) causes a steady realease of energy.
    $$\Delta U = \Delta QV$$
    Better to discess the rate, which is really known as:
    $$\text{Power}=P=\frac{\Delta U}{\Delta t} = \frac{\Delta U}{\Delta t} V = IV$$
    $$\boxed{$$P = IV = I^2R = \frac{V^2}{R}$$}$$
    This power is also known as Joule heating.
    \paragraph{Putting this into practice:}many circuits can be analyzed with just Ohm's Law and Resistance.
    \begin{itemize}
        \item What is the total power delivered? 
        \item What is the power dissipated in each R?
    \end{itemize}
    \begin{circuitikz}\draw
        (0,-2)to[battery, l=$12V$] (0,1)
        (0,1) to (3,1)
        (3,1)to[resistor, l=$2\Omega$](3,-1)
        (2, -1)to(4,-1)
        (2,-1)to[resistor, l=$3\Omega$](2,-3)
        (4,-1)to[resistor, l=$6\Omega$](4,-3)
        (2,-3)to(4,-3)
        (3,-3)to(3,-4)
        (3,-4)to(0,-4)
        (0,-4)to(0,-2)
    ;\end{circuitikz}
    $P_{\text{dissapated in} 3\Omega} = \frac{V_3^2}{R} = \frac{(6V)^2}{3\Omega}=12W$
    \subsection{Direct Current Circuits}
    Real battery is an ideal $\calE$MF plus intended resistance\\
    \begin{circuitikz}\draw
        (0,0)
        node[label={left:Batteries are a source of voltage}] {}
        to[battery, l=$+$](2,0)
    ;\end{circuitikz}
    \\Source of voltage = "electromotive force" = $\calE$MF= $\calE$
    $$\boxed{$$V = \calE - Ir$$}$$
    "Open-circuit voltage", where $I=0$
\end{document}